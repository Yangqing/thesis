\chapter{Conclusion}

In this thesis I have analyzed two key components in the computer vision research: to learn better image features with solid theoretical justifications, and to re-visit the existing vision problem statement to a more practical and human-like one. To this end, I have proposed and analyzed novel receptive field learning and dictionary learning methods, which is justified by the \nystrom sampling theory, that learns more compact and effective features for object recognition tasks. Having analyzed the feature generation pipeline, I then move to the more high-level scrutiny of the current object recognition experimental setting. By combining the otherwise independently developing computer vision and cognitive science studies, I have presented the first large-scale system that allows computers to learn and generalize closer to what a human learner will do. I have also collected a large-scale human behavior database, which would hopefully enable further research along this research direction.

It is both surprising and thrilling to witness the recent success, or rediscovery as we may argue, of convolutional neural networks in the object recognition research. I have devoted the last part of my thesis to making a better engineering system for deep learning research, as well as providing an extensive analysis on how features learned from the state-of-the-art CNN pipeline would serve as a general-purpose visual descriptor that could be adopted in various applications. Computer vision research has always been pushed by great open-source vision libraries, and it is my sincere hope that my contribution would boost this new field in the coming years.