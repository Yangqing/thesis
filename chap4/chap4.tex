\chapter{Latent Task Adaptation with Concept Hierarchy}

While the concept learning model proposed above has shown promising performance in modeling human performance, one may ask the question ``how this would benefit real-world applications?''. This section is devoted to present a real world application - learning multiple specific classification tasks - and demonstrates how a concept learning framework would allow much more flexible classifiers to be deployed.

\section{Introduction}
Recent years have witnessed a growing interest in object classification tasks involving specific sets of object categories, such as fine-grained object classification \cite{farrell2011birdlets, khosla2011novel} and home object recognition in visual robotics. Existing methods in the literature generally describe algorithms that are trained and tested on exactly the same task, \ie we assume the training data and testing data share the same set of object labels. A dog breed classifier is trained and tested on dogs and a cat breed classifier done on cats, without the use of out-of-task images.

\begin{figure}
    \centering
    \newcommand{\introim}[1]{\includegraphics[height=0.15\linewidth]{figs/taskadaptation/thumbnails_sub/#1.JPEG}}
    \newcommand{\introsep}{\hspace{0.005\linewidth}}
    \newcommand{\demoim}[1]{\includegraphics[height=0.18\linewidth]{figs/taskadaptation/thumbnails_sub/#1.JPEG}}
    \begin{minipage}{0.5\linewidth}
    \includegraphics[width=1.\textwidth]{figs/taskadaptation/ilsvrc_graph_stitched.pdf}
    \end{minipage}\hfill
    \begin{minipage}{0.45\linewidth}
        \begin{flushleft}
        % feline
        \introim{34544}\introsep
        \introim{82396}\introsep
        \introim{66068}\introsep
        \introim{36749}\introsep
        \introim{67538}\\
        \introim{19914}\introsep
        \introim{22979}\introsep
        \introim{101078}\introsep
        \introim{45885}\vspace{0.05in}

        %dogs
        \introim{62437}\introsep
        \introim{90868}\introsep
        \introim{84585}\introsep
        \introim{12176}\introsep
        \introim{83447}\\
        \introim{16261}\introsep
        \introim{111956}\introsep
        \introim{57158}\introsep
        \introim{26210}\vspace{0.05in}

        %vehicle
        \introim{101957}\introsep
        \introim{111522}\introsep
        \introim{47307}\introsep
        \introim{5323}\\
        \introim{87378}\introsep
        \introim{11431}\introsep
        \introim{104223}\introsep
        \introim{137726}
        \end{flushleft}
    \end{minipage}
    \caption{Visualization of the ImageNet hierarchy as a tree structure, with three subtrees corresponding to dogs, feline and vehicles highlighted in color. Exemplar images from these three subtrees are presented on the right. Such specific subtrees usually correspond to classification tasks of interest (``which breed of dog is this?'').}\label{fig:ilsvrc2010tree}
    \newcommand{\exampleim}[1]{\includegraphics[height=0.18\linewidth]{figs/taskadaptation/thumbnails_sub/#1.JPEG}}
      \begin{tabular}{ccc}
      \exampleim{148153} & \exampleim{95105} & \exampleim{107750}\\
      {\bfseries golden retriever} & {\bfseries tabby cat} & {\bfseries garbage truck}\\
      (ice bear) & (dungeness crab) & (boathouse)
      \end{tabular}
      \caption{Adapting the ImageNet classifier allows us to perform accurate prediction (bold), while the original classifier prediction (in parentheses) suffers from a higher confusion. Note that the classification is carried out together with a set of other images as the task context.}\label{fig:tasks}
\end{figure}

However, two observations may render this ``one (multi-class) classifier per task'' approach suboptimal. First, it's known that using images of related tasks is often beneficial to build a better model for the general visual world \cite{raina2007self}, which serves as a better regularization for the specific task as well. Second, object categories in the real world are often organized in, or at least well modeled by, a nested taxonomical hierarchy (\eg Figure \ref{fig:ilsvrc2010tree}), with classification tasks corresponding to intermediate subtrees in this hierarchy, and recent efforts on the ImageNet challenge \cite{ilsvrc,lin2011large,sanchez2011high,krizhevsky2012imagenet} have leveraged the use of large-scale data to learn such information. While it is reasonable to train separate classifiers for specific tasks, this quickly becomes infeasible as there are a huge number of possible tasks - any subtree in the hierarchy may be a latent task requiring one to distinguish object categories under the subtree.

Thus, it would be beneficial to have a system which learns a large number of object categories in the world, and which is able to quickly adapt to specific incoming classification tasks (subsets of all the object categories) once deployed. We are particularly interested in the scenario where tasks are not explicitly given, but implicitly specified with a set of query images, or a stream of query images in an online fashion. This has practical importance: for example, one may want to have a single mobile app that adapts to plant recognition on a field trip after a few image queries, and that shifts to grocery recognitions when one stops by the grocery store. This is a new challenge beyond simple classification - one needs to discover the latent task using the context given by the queries, a problem that has not been tackled in previous classification problems.

It turns out that this problem is inherently similar to the concept learning problem that we focused on in the previous chapter: while classifying a set of images, one could consider this image set as examples of a latent ``task'', or ``concept'', that corresponds to the current application scenario. Thus, in addition to identifying the latent concept itself, which is of interest to visual concept learning, the additional problem is to perform classification under this concept to reveal more fine-grained category labels (such as different species of dogs and birds). This is perfectly applicable under the visual concept learning framework. In this chapter, we will demonstrate one system that achieves this ``learn big, predict specific'' goal.

Regarding related works along the task adaptation idea, the problem of task adaptation is analogous to, but essentially distinctive from domain adaptation \cite{saenko2010adapting,kulis2011you}. While domain adaptation aims to model the \emph{perceptual} difference of the training and testing images from the same labels, task adaptation focuses on modeling the \emph{conceptual} difference: different label spaces during training and testing. Additionally, as one is often able to use large amounts of data during training, we assume that the testing tasks involve subsets of labels encountered during training time.

%Predicting the intermediate concept in a hierarchy with a set of examples has been discussed in psychology \cite{xu2007word,abbottconstructing,tenenbaum2001generalization}. These methods often make a simplified assumption that labels (leaf nodes in the hierarchy) are given for the input images. We believe our paper is the first to connect such psychological study with computer vision research by directly taking perceptual inputs, allowing one to perform generalization with images of unknown category.

There are several algorithms in image classification that use label hierarchy or structured regularizations to learn better classifiers \cite{salakhutdinov2011learning,harchaoui2012large,gao2011discriminative}, or to leverage the accuracy and information gain from classifiers \cite{deng2012hedging}. These methods still assume an identical label space for training and testing. The ultimate goal thus remains to be better accuracy on classifying individual images, not to adapt to different tasks during testing time by utilizing contextual information. Better classifiers presented in these papers could, of course, be incorporated in our model to improve the end-to-end performance of task adaptation.

Finally, it is known that contextual information, such as scene context and co-occurring context within a image, could be adopted for better detection \cite{torralba2003contextual} or scene understanding \cite{li2009towards}. In this paper we utilize a novel type of context - \emph{task context} - that is implied by a semantically related group of images.


\section{Problem Statement}
For the sake of clarity, we will first state the task adaptation problem using notations from the previous chapter, and then highlight the connection between the concept learning problem and the latent task adaptation problem.

Formally, we define a classification \emph{task} to be a subset of all the possible object labels that are semantically related (such as all breeds of dogs in ImageNet). During training time, the computer is given all the training images from all these classes, and it will learn one single multi-class classifier. During testing time, a number of query images are randomly sampled from the labels belonging to a task, and the learning algorithm needs to give predictions on these images.

This scenario is much differnt from the conventional image classification problem setting, as being used in various benchmarks such as Caltech-101 \cite{fei2006one} and ILSVRC \cite{ilsvrc10}. Conventionally, we assume a set of mutually exclusive class labels to be presented during both training and testing time. From a probabilistic perspective, it means that the test images are assumed to be drawn i.i.d.\ from the same label distribution as the training images are. In our problem setting, testing images are mutually related since they together define the task. This makes more practical sense, since one may expect a computer agent to utilize its environment to preform better classification. For example, one could switch to classifying grocery items in a grocery store, and to classifying different animals during a zoo visit. It would be extremely unlikely (and not preferred) for an item in a grocery store to be a giraffe, given the context information.

As stated in the previous section, we are interested in the scenario when the task is \emph{latent}, \ie only implicitly specified by a set of test images. We introduce two key components for modeling the generative process of test images: a latent task space that defines possible tasks and their probability, and a procedure to sample images given a specific latent task. Specifically, we propose the graphical model in Figure \ref{fig:conceptgraph} which generates a set of $N$ test images when given $T$ possible tasks and $K$ object categories:
\begin{enumerate}
  \item Sample a latent task $h$ from the task priors $P(h)$ with hyperparameter $\balpha$;
  \item For the $N$ images:
    \begin{enumerate}
        \item Sample an object category $y_i$ from the conditional probability $P(y_i|h; \bbeta_{h})$;
        \item Sample an image from category $y_i$ with $P(x_i | y_i; \btheta_{y_i})$.
    \end{enumerate}
\end{enumerate}
where the parameters $\balpha, \bbeta, \btheta$ are defined as in the previous chapter.

\begin{figure}
  \centering
  \includegraphics[width=0.15\textwidth]{figs/taskadaptation/our_model_vertical.pdf} \hspace{0.25in}%
  \caption{The generative model for the latent task and corresponding query images.}\label{fig:conceptgraph}
\end{figure}



\section{Efficient Learning and Inference}\label{sec:algo}
The probabilistic model involves multiple parameters to be estimated and nested hidden variables during the inference phase. In this section, we present a novel approach to estimate the confusion matrix for the classifier, and a linear-time inference algorithm that jointly identifies the latent task and predictions for individual images.

\subsection{Linear Time MAP Inference}
A conventional way to do probabilistic inference with nested latent variables is to use variational inference or Gibbs sampling to find a lower bound of the posterior probability. This, however, may involve multiple iterations over the hidden variables and may be slow. We show that when the latent task space is organized in a DAG structure, the exact MAP solution (Eqn.\ (\ref{eqn:MAP})) could be found with an efficient dynamic programming algorithm that has complexity linear to the number of possible tasks.

We first note that the logarithm of posterior probability in Eqn.\ \ref{eqn:MAP} could be expanded as
\begin{equation}
\log P(h, \mathcal{Y} | \mathcal{X}) \propto \log\alpha_h + \sum\nolimits_{i=1}^{N}\log(\beta_{hy_i}C_{y_if(\bx_i)}).
\end{equation}
Notice that the size constraint defining the latent task space gives us $\beta_{hy_i} = \frac{1}{|h|}I(y_i \in h)$, the equation above could further be written as
\begin{equation*}
    \log\alpha_h - N \log|h| + \sum\nolimits_{i=1}^{n} (\log C_{y_if(\bx_i)} + \log I(y_i\!\in\!h)),
\end{equation*}
where one can observe that $h$ and $\mathcal{Y}$ decouples except for the $I(y_i\in h)$ term. As the latent tasks are organized as a tree-based hierarchy, we can define auxiliary functions
\begin{equation}
q_i(h) = \max_{\mathcal{Y}} \left[\log C_{y_if(\bx_i)} + \log I(y_i\in h)\right],
\end{equation}
which could be computed recursively as
\begin{equation}
q_i(h) = \max\nolimits_{\{h' \in child(h)\}} q_i(h'),
\end{equation}
where $child(h)$ is the set of children of $h$ in the tree. Finally, the latent task could be estimated as
\begin{equation}\label{eqn:taskargmax}
    \hat{h} = \argmax{h} \big[\log(\alpha_h) - N\log|h| + \gamma\sum\nolimits_{i=1}^{N} q_i(h)\big],
\end{equation}
and the corresponding $\hat{y}_i$s could be identified by taking the $\mathrm{argmax}$ of the corresponding $q_i(h)$.

We note that we added a hyperparameter $\gamma$ in the equation above. In practice, simply finding the MAP solution (using $\gamma=1$) often involves a task that is smaller than the ground truth, as there are two ways to explain the predicted labels: assuming correct prediction and a task of larger size, or assuming wrong prediction and a task of smaller size. The latter is preferred by the size principle, especially for classes with low classification accuracy. We found it beneficial to explicitly add a weight term that favors the classifier outputs using $\gamma>1$ learned on validation data.

In general, our dynamic programming method runs in $O(T\!Nb)$ time where $T$ is the number of tasks, $N$ is the number of query images, and $b$ is the branching factor of the tree (usually a small constant factor). This complexity is linear to the number of testing images and to the number of latent tasks, and is usually negligible compared to the basic classification algorithm, which runs $O(K\!N\!D)$ time where $K$ is the number of classes and $D$ is the feature dimension (usually very large).

%\subsection{Online Task Discovery and Classification}
Finally, one may prefer an online algorithm that could take new images as a stream, performing classification sequentially while discovering the latent task on the fly. We note that our method could be easily adapted to this end. Specifically, $q_i(h)$ serves as the sufficient statistics for the task discovery, and we only need to keep record of the accumulated auxiliary function values seen so far as 
\begin{equation}
    q_{:n}(h) = \sum\nolimits_{i=1}^{n-1} q_i(h)
\end{equation}
for the $n$-th image for each task candidate $h$. This allows us to perform online classification with $O(M)$ memory without storing the full history of images.


\section{Experiments}
We conduct our experiment on the ILSVRC 2010 dataset \cite{ilsvrc}, where both validation and test data are available. For all the experiments, we learn the parameters of the model on the training and validation data, and report the performance on the test images.

We note that more comprehensive features and better classification pipelines may lead to better 1-vs-all accuracy on ImageNet, but it is not the main goal of the paper, as we focus on the adaptation on top of the base classifiers. Recent efforts on learning better classifiers, such as the ones presented in \cite{sanchez2011high,krizhevsky2012imagenet} could be seamlessly incorporated into our learning framework for general performance increases.

\subsection{Estimating the Confusion Matrix}\label{sec:expconfmat}
As stated in Section \ref{sec:algo}, an good estimation of the confusion matrix $\bC$ is crucial for the probabilistic inference. We evaluate the quality of different estimations using the test data: for each testing pair $(y, \hat{y})$, where $\hat{y}$ is the classifier output, its probability is given by the confusion matrix entry $C_{y\hat{y}}$. The perplexity measure \cite{jurafsky2000speech} then evaluates how ``surprising'' the confusion matrix sees the testing data results (a smaller value indicates a better fit):
$$
perp = \mathrm{Power}\Big(2, \big(\sum\nolimits_{i=1}^{N_{te}}\log_{2}C_{y_i\hat{y}_i}\big)/N_{te}\Big),
$$
where $N_{te}$ is the number of testing images. Overall, we obtained a perplexity of 46.27 using our unlearning algorithm, while the validation data gave a value of 68.36 and the training data (without unlearning) gave 94.69, both worse than our unlearning algorithm. We refer to the supplementary material for a more complete analysis of the performance of different methods.

% In the supplementary material we also visualize the confusion matrix entries that are non-zero for the testing data, but incorrectly predicted as zero by the methods. Overall, matrices estimated from the training and validation data both yield a large proportion ($>$70\%) of incorrect zero entries due to overfitting and lack of validation images respectively, while our method gives a much better estimation with incorrect zero entries $<$25\%.

%We evaluated multiple choices of the confusion matrix sources: using the ones obtained from the training and validation data respectively, and the one we obtained from one-step unlearning. 

%Figure \ref{fig:conftable} visualizes the distribution of zero-entries in the estimated confusion matrix, where the testing result has non-zero values. Matrices estimated from the training and validation data both yield a large proportion ($>$70\%) of incorrect zero entries, due to overfitting and lack of validation images respectively. In contrast, our method gives a much denser estimation (with incorrect zero entries $<$25\%), with the remaining sparsity further fixed by the smoothing step.

\subsection{Adapting Classifiers with Known Tasks}\label{sec:knowntask}
An important question to ask is, even if we are allowed to retrain task-specific classifiers, do we want to do the retraining? We first analyze the benefits of retraining versus our adaptation method. To this end, we specify 5 subtrees from the ILSVRC hierarchy: {\texttt building}, {\texttt dogs}, {\texttt feline} (the superset of cats), {\texttt home appliance}, and {\texttt vehicle}, the subcategories of which are often of interest. Figure \ref{fig:tasks} visualizes the corresponding subtrees for dog, feline and vehicles respectively. We explicitly trained classifiers on these three subtrees only, and compared the retrained accuracy against our adapted classifier with the given task. We also test the naive baseline that uses the raw 1000 class predictions, and the forced choice baseline (FC) which simply selects the class under the task that has the largest output from the original classifiers. Table \ref{tab:knowntask} summarizes the performance of the algorithms.

It is worth pointing out that retraining the classifiers for the specific tasks does \emph{not} help improve the classification accuracy, although retraining requires additional nontrivial computation cost. On contrary, it is always helpful to use out-of-task data to train a larger classifier and then take the subset with forced choice. One possible explanation is that this gives us more information about the general image statistics (similar to a better regularization term). Our method further benefits from the statistics from all the classifiers (for in-task and out-of-task classes) in the proposed probabilistic framework to achieve the best adapted accuracy in most cases (only slightly worse than the FC baseline on {\texttt vehicle}).

\subsection{Joint Task Discovery and Classification}

\begin{table}
    \centering
    \small
    \begin{tabular}{c|cccc}
        \hline \hline
        Task & Naive & Retraining & FC & Ours \\
        \hline
        {\texttt building} & 55.48 & 78.67 & 81.48 & {\bfseries 82.19} \\
        {\texttt dog} & 35.37 & 39.94 & 42.95 & {\bfseries 43.76}\\
        {\texttt feline} & 47.13 & 61.07 & 62.67 & {\bfseries 63.54}\\
        {\texttt home app} & 50.78 & 67.30 & 69.26 & {\bfseries 70.52} \\
        {\texttt vehicle} & 55.62 & 61.43 & {\bfseries 63.41} & 63.28\\
        \hline \hline
    \end{tabular}
    \caption{Classification accuracy on given tasks (subtrees) of the whole ILSVRC data. See subsection \ref{sec:knowntask} for details.}\label{tab:knowntask}
\end{table}

\begin{table}
\small
    \centering
    \begin{tabular}{c|cc|cc}
        \hline \hline
        \multirow{2}{*}{Method} & \multicolumn{2}{c|}{query size=5} & \multicolumn{2}{c}{query size=100}\\
        \cline{2-5}
        & $s(h, \hat{h})$ & Accuracy &  $s(h, \hat{h})$ & Accuracy \\
        \hline
        Naive   & 1.54 & 42.75 & 1.50 & 42.68 \\
        Proto   & 8.14 & 43.16 & 60.39 & 50.28 \\
        Hist    & 22.21 & 44.84 & 96.61 & 59.87 \\
        Hedging & 39.12 & 44.81 & 50.34 & 51.83 \\
        Ours    & {\bfseries 84.43} & {\bfseries 65.89} & {\bfseries 99.37} & {\bfseries 70.70} \\
        \hline \hline
        Oracle  & 100.0 & 70.36 & 100.0 & 70.88 \\
        \hline \hline
    \end{tabular}
    \caption{The average task overlap score and the average accuracy for the algorithms, under query sizes 5 and 100 respectively. All numbers are in percentage. The last row provides the oracle performance in which the ground truth task is given.}\label{tab:jointclassify}
\end{table}

\begin{figure}
    \centering
    \includegraphics[width=0.235\textwidth]{figs/taskadaptation/offline_accuracy.pdf}%
    \includegraphics[width=0.235\textwidth]{figs/taskadaptation/offline_overlap_score.pdf}
    \caption{Classification accuracy (left) and the task overlap score (right) with different query set sizes for our method and the baselines.}\label{fig:jointclassify}
\end{figure}

\definecolor{naive}{rgb}{0.491, 0.331, 0.651}
\definecolor{proto}{rgb}{0.967, 0.530, 0.135}
\definecolor{hist}{rgb}{0.418, 0.621, 0.292}
\definecolor{hedge}{rgb}{0.894, 0.122, 0.129}
\definecolor{adapt}{rgb}{0.167, 0.502, 0.693}
\begin{figure}
    \centering
    \small
    %\begin{small}
    \setlength\fboxsep{1.5pt}
    \setlength\fboxrule{0pt}
    \newcommand{\testim}[1]{\fbox{\begin{minipage}[c]{0.095\textwidth}\includegraphics[width=1.\textwidth,height=1.\textwidth]{figs/taskadaptation/thumbnails_sub/#1.JPEG}\end{minipage}}}
    \newcommand{\tasks}[5]{\multirow{4}{*}{\parbox{0.12\textwidth}{%
                Predicted task:\\%
                \textcolor{naive}{#1}\\%
                \textcolor{proto}{#2}\\%
                \textcolor{hist}{#3}\\%
                \textcolor{hedge}{#4}\\%
                \textcolor{adapt}{#5}\\%
    }}}
    \newcommand{\task}[1]{\parbox{0.085\textwidth}{\raggedleft Task:\\#1}}
    \begin{tabular}{r|ccccc|c}
        \hline\hline
        \task{{\bfseries kitchen app}} & \testim{102835} & \testim{19992} & \testim{89217} & \testim{32216} & \testim{123376} & \tasks{entity}{artifact}{artifact}{consumer goods}{kitchen app}\\
        Label & ice maker & espresso maker & primus stove & Dutch oven & ice maker & \\
         Ours & electric range & espresso maker & primus stove & Dutch oven & ice maker & \\
     Baseline & bookcase & web site & carpenter's kit & snail & scanner & \\
        \hline\hline
        \task{{\bfseries toiletry}} & \testim{100057} & \testim{40023} & \testim{78466} & \testim{48008} & \testim{59007} & \tasks{entity}{entity}{entity}{instrumentality}{toiletry}\\
        Label & lipstick & face powder & nail polish & lotion & hair spray & \\
         Ours & lipstick & face powder & nail polish & lotion & hair spray & \\
     Baseline & toothbrush & dune & bath towel & vending machine & military uniform & \\
        \hline\hline
        \task{{\bfseries woodwind}} & \testim{125599} & \testim{11587} & \testim{146326} & \testim{10043} & \testim{4491} & \tasks{entity}{artifact}{artifact}{device}{reed instrument}\\
        Label & bassoon & flute & sax & oboe & sax & \\
         Ours & bassoon & bassoon & sax & oboe & sax & \\
     Baseline & harp & prison & sax & fountain pen & turban & \\
        \hline\hline
        \task{{\bfseries game}} & \testim{19441} & \testim{124828} & \testim{64107} & \testim{109738} & \testim{139256} & \tasks{entity}{living thing}{entity}{chordate}{game}\\
        Label & ptarmigan & partridge & pheasant & black grouse & quail & \\
         Ours & ptarmigan & partridge & pheasant & black grouse & black grouse & \\
     Baseline & giant panda & orchid & Komodo dragon & Border collie & Newfoundland & \\
        \hline\hline
    \end{tabular}
    \caption{Exemplar classification results where incorrect labels are predicted by the base classifiers, but are corrected by our method that benefits from identifying the latent task. Each row shows 5 images from a latent task, and on the right we give the predicted task by different algorithms, ordered and colored as \textcolor{naive}{naive}, \textcolor{proto}{proto}, \textcolor{hist}{hist}, \textcolor{hedge}{hedge}, and \textcolor{adapt}{adapt}. The ground truth label, our prediction and the original classifier's output are provided for each image.}
\end{figure}


We next analyze the performance when we have the classifier trained on the whole ILSVRC data, and adapt it to an unknown task that is defined by a set of query images. The forced choice option is not available in this case as we do not know the latent task beforehand, and one has to use the semantic relationships between the query images to infer the latent task.

To sample the latent tasks, we used the Erlang prior defined in Section \ref{sec:model} from the ImageNet Tree excluding leaf nodes (as leaf nodes would contain only 1 label). We then randomly sampled $N$ query images from the subtree of the sampled task. All query images were randomly selected from the test images of ILSVRC and had not been seen by the classifier training. We varied the value $N$ to assess the quality of task discovery under different set sizes. For each query image size $N$, we created 10,000 independent tasks and reported the average performance here.

To evaluate the goodness of the inferred latent task and the accuracy, we compute the overlap between the ground truth task $h$ and the predicted task $\hat{h}$ as
\begin{equation}
    s(h, \hat{h}) = |h\cap \hat{h}| / |h \cup \hat{h}| \times 100 \%,
\end{equation}
where $\cap$ and $\cup$ are the intersection and union operations on sets, and $|\cdot|$ denotes the size of a set. For each task, we then compute the accuracy with the predicted labels $\hat{\mathcal{Y}}$ in the standard classification evaluation way. We then report the averaged overlap score and averaged per-task prediction accuracy.

To the best of our knowledge there is no published classification algorithm that is able to identify the latent task, \ie the intermediate node in the taxonomy hierarchy, given a \emph{set} of query images. Thus, we compare our algorithm against the following baselines that are natural extensions from conventional classification methods:
%\begin{enumerate}\setlength{\itemsep}{0pt}\setlength{\parskip}{0pt}\setlength{\leftmargin}{0pt}
\begin{list}{\labelitemi}{\leftmargin=1em}\setlength{\itemsep}{0pt}\setlength{\parskip}{0pt}
    \item {\bfseries Naive approach}: simply taking the class with the highest prediction score from all the ILSVRC classes.
    \item {\bfseries Prototype approach}: we use the conditional probability $p(y|h)$ as a vector for each task $h$, and use the task that yields the smallest average distance to each query image (using the classifier outputs) as the predicted latent task. Classification is then performed under this predicted task.
    \item {\bfseries Histogram approach}: similar to the prototype approach, but instead of computing pairwise distances to individual query images, we select the task $h$ that yields the smallest $\chi^2$ distance between $p(y|h)$ and the histogram of predictions averaged over all queries.
    \item {\bfseries Hedging approach}: we extend the hedging idea \cite{deng2012hedging} to handle sets of query images. Specifically, we find the intermediate node that maximizes the information gain while maintaining an overall accuracy above a threshold $\epsilon$ over the set of query images. The corresponding task is then chosen as the predicted latent task. We tune the threshold $\epsilon$ on the validation data so that the averaged per-task accuracy is maximized.
%\end{enumerate}
\end{list}
We also test an oracle model, in which we explicitly tell the classifier the latent task and perform classification on the subset of labels with the task ground truth. This serves as an upper bound of all methods above, and helps us understand how well different algorithms perform. Regarding the classifier outputs, we used the soft output from the logistic regression for our method, and choose between the soft output and 0-1 hard output for the baseline methods, reporting the better performance of the two here.

%\footnote{As a minor note, the hedging method works well with soft outputs, while the prototype and histogram methods prefer soft outputs when the query size is small, and hard outputs when large.}.

Table \ref{tab:jointclassify} summarizes the performance of the methods above with a small query set size (5 images) and a relatively large size (100 image). Further, Figure \ref{fig:jointclassify} shows the performance when we vary the size from 1 to 500. It could be observed that when we have a reasonable amount of testing queries, identifying the latent task leads to a significant performance gain than the baseline method that does classification against all possible labels, with an increase of near 30\% percent. Even with a small query size (such as 5), the performance gain is already noticably high, indicating the ability of the algorithm to perform task adaptation with very few images from the latent task.

\subsection{Online Evaluation}\label{subsec:online}
Our final evaluation tests the performance of the proposed method in an online fashion - when images of an unknown task come as a streaming sequence. Intuitively, our algorithm obtains better information about the unknown task as new images arrive, which would in turn increase the classification accuracy. We test such conjecture by evaluating the averaged accuracy of the $n$-th image, over multiple independent test query sequences that are generated in the same way as described in the previous subsection.

Figure \ref{fig:online} shows the average accuracy of the $n$-th query image, as well as the overlap between the identified task so far and the ground truth task. With the joint probabilistic inference, we obtain a significant performance increase after only a few images. This has particular practical interest, as one may want the computer to quickly adapt to a new task / environment with only a small number of queries. It is worth pointing out that with heuristic task estimation methods (see the baselines in Figure \ref{fig:online} left), one may incorrectly assert the latent task, which then hurts classification performance for the first few query images.

\begin{figure}
    \centering
    \includegraphics[width=0.24\textwidth]{figs/taskadaptation/online_accuracy.pdf}%
    \includegraphics[width=0.24\textwidth]{figs/taskadaptation/online_overlap_score.pdf}
    \caption{Classification accuracy (left) and task overlap score (right) of our online algorithm against baselines. See subsection \ref{subsec:online} for details.}\label{fig:online}
\end{figure}
