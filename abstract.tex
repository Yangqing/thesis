\begin{abstract}

In my thesis I present my work towards learning a better computer vision system that learns and generalizes object categories better, and behaves in ways closer to what human learners do. Specifically, I will focus on two key components of such a system: learning better features, and revisiting existing problem statements. For the first part, I will propose and analyzed novel receptive field learning and dictionary learning methods, which is justified by the \nystrom sampling theory, that learns more compact and effective features for object recognition tasks. For the second part, I will combine the otherwise independently developing computer vision and cognitive science studies, and present the first large-scale system that allows computers to learn and generalize closer to what a human learner will do. I will also provide a large-scale human behavior database, which would hopefully enable further research along this research direction.

Following the recent success of convolutional neural networks, I will present and release a well-engineered framework for general deep learning research, and provide an extensive analysis on the generality of deep features learned from the state-of-the-art CNN pipeline: whether they serve as a general-purpose visual descriptor that could be adopted in various applications, and future research directions made possible by such general features.

\end{abstract}
